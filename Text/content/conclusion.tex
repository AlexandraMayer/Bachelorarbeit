\chapter{Fazit und Ausblick} \label{Fazit und Ausblick}
In dieser Arbeit habe ich empirische Beobachtungen an drei großen Haltestellen der Münchner U-Bahn durchgeführt. Dabei habe ich Aufnahmen von Fahrgästen angefertigt, die ich zur Datenextraktion verwendet habe. Die von mir extrahierten Daten wurden dann in \texttt{Jupyter Notebooks} weiterverarbeitet und in Bezug auf den Fahrgastwechsel analysiert. Zum Beispiel kam ich zu dem Ergebnis, dass die Fahrgastwechselzeit der am Fahrgastwechsel beteiligten Personen aus meinen Daten durch lineare Regression abgeschätzt werden kann. Zudem wurde durch die Analyse der Daten klar, dass die Verhaltensweise "`im Weg Stehen"', wie zuvor vermutet, einen zeitlichen Einfluss besitzt. In den Daten wurde auch ein Merkmal identifiziert, dass einen signifikanten zeitlichen Einfluss besitzt. Das Merkmal "`sperrig"' trägt eine Person, wenn sie mehr Platz einnimmt als andere Personen. Eine Person ist \zB "`sperrig"', wenn sie ein großes Gepäckstück mit sich führt. Diese Art von Personen tritt an 46.43 \% der Türen auf und verlangsamt den Fahrgastwechsel, da durch sperrige Personen Fahrgäste nicht mehr zu zweit durch die Tür gehen können. Zudem ging aus meinen Daten hervor, dass es nicht nur Einsteiger und Aussteiger gibt. Es gibt es noch den sogenannten Prozesstypen Platzmacher. Dieser Prozesstyp beschreibt Fahrgäste, die nur aus dem Zug aussteigen, um Aussteigern hinter ihnen Platz zu machen. Nachdem diese Personen Platz gemacht habe steigen sie wieder in den Zug ein. Dieses Verhalten wurde an 16.07 \% der Türen gezeigt. Der Prozesstyp Platzmacher wurde als relevant eingestuft, da diese Personen sowohl aus- als auch einsteigen und somit einen signifikanten Einfluss auf die Fahrgastwechselzeit besitzen. Zudem ergab sich aus den Daten, dass es Typen von Personen gibt. Diese Typen werden auf Grund ihres Verhaltens als defensiv, aggressiv oder normal bezeichnet. Der Anteil der aggressiven und defensiven Personen ist hierbei so signifikant, dass ich ihr Verhalten im Modell umgesetzt habe. So wurde in meinem Modell zum Beispiel dargestellt, dass defensive und normale Einsteiger "`früher Einsteigen"', also einsteigen, wenn noch Aussteiger den Zug verlassen, während defensive Personen immer warten bis alle Aussteiger aus dem Zug ausgestiegen sind. Auch bei der Platzsuche unterscheiden sich die unterschiedlichen Typen. Während defensive Typen sich immer hinter die letzte Person im "`Wartebereich"' stellen, suchen normale nach dem nächsten freien Platz und aggressive stellen sich in den Weg, wenn vorne im Wartebereich kein Platz mehr ist. Ein weiteres Verhalten von defensiven Typen setzte ich ebenfalls durch kognitive Heuristik um. Defensive Typen betreten den Türbereich erst, wenn die Person vor ihr nicht mehr in diesem ist, sie warten also darauf, dass sie alleine durch die Tür gehen können. Eine weitere Unterscheidung der Typen, liegt darin, wie lange Aussteiger mit einem bestimmten Typ nach der Türöffnung warten, bevor sie mit dem Aussteigen beginnen. Dies wurde durch Intervalle für Startzeiten dargestellt, wobei Aggressive am kürzesten und Defensive am längsten warten bevor sie aussteigen.

Vergleicht man die bisherige Modellierung von accu:rate mit den gesammelten Daten, ist zu bemerken, dass dieses den absoluten Standardfall des Fahrgastwechsels ganz gut darstellen kann. In der Modellierung wird die Entscheidungsfindung der einzelnen Personen im Prozess jedoch nicht beachtet, was mein Modell verbessern soll. Ob dieses Modell jedoch wirklich zu einem besseren Ergebnis führt, wurde im Zuge dieser Bachelorarbeit jedoch nicht mehr getestet. 

Da das Modell nicht implementiert wurde, konnte ein Vergleich des Modells mit der Modellierung von accu:rate über eine Simulation nicht durchgeführt werden. Außerdem ist dadurch nicht klar, ob das Modell einen realistischen Fahrgastwechsel darstellen kann, da eine Validierung des Modells nicht durchgeführt wurde. Es wäre also interessant, dieses in eine bereits existierende Software für Fußgängersimulationen, \zB crowd:it, einzufügen, um das erstellte Modell zu Kalibrieren und Validieren, um es möglicherweise zukünftig in Planungen der MVG einsetzen zu können.

Für einige Auswertungen sind nur wenig Daten vorhanden. Durch diese geringe Datenlage, vor allem für Platzmacher und der ersten Person, die aussteigt und angibt, wie lange ein Typ vor dem Aussteigen wartet, könnte es sein, dass manche Auswertungen fehlerhaft sind. Dies könnte jedoch auch durch die Umsetzung des Modells in Software, oder durch weitere Beobachtung geprüft werden.

Wird das Modell umgesetzt und validiert, wäre es zudem noch interessant, das Modell für den Fahrgastwechsel in den Kontext eines Bahnsteiges einzufügen. Besonders interessant wäre die Kombination mit anderen Simulationsmodelle für das Innere des Zuges oder des Bahnsteigs. Diese könnten Modelle sein, die darstellen:
\begin{itemize}
\item Wie Personen einen Sitzplatz im Zug wählen (\cite{Schottl.2016} oder sich entscheiden stehen zu bleiben.
\item Wie Aussteiger sich an der Tür anstellen, um den Zug zu verlassen.
\item Wie Personen sich auf dem Bahnsteig verhalten (\cite{Davidich.2013}, \cite{Chen.2017})
\end{itemize}

Wie Einsteiger den Wagon und die Tür wählen, an der sie Einsteigen wollen, wird in meinem Modell vernachlässigt. Auch dieser Entscheidungsprozess könnte für mein Modell interessant sein, wenn nicht nur eine Tür, sondern ein gesamter Zug modelliert werden soll. Zudem erkannte ich auf den Aufnahmen auch Personen die sich zunächst an einer Tür "`anstellten"', sich dann aber doch dafür entschieden, an eine andere zu gehen. Wann dieses Verhalten auftritt könnte für die Simulationen eines Zuges interessant sein.

Das von mir erstellte Modell enthält alle als relevant erkannten Verhaltensweisen und Merkmale und könnte somit zu einem besseren Fahrgastwechsel-Modell führen, dass in der Planung von neuen Stationen oder Fahrplänen eingesetzt werden könnte. Aufgrund der geringen Datenlage und der Tatsache, dass das Modell nicht implementiert und somit nicht validiert wurde, ist es jedoch notwendig diese Implementierung nachzuholen und das Modell zu testen, bevor es eingesetzt werden kann.