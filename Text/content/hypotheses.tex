\chapter{Forschungsfragen} \label{Forschungsfragen}
Um eine Entscheidung darüber zu treffen, welche Daten des Fahrgastwechsels aus den Videos erhoben werden sollen, verfeinere ich Forschungsfragen aus Abschnitt \ref{Stand der Technik}. Im Folgenden wiederhole diese Fragen und nenne jeweiligen Verfeinerungen, falls vorhanden.
\begin{enumerate}
 \item "`Kann die Fahrgastwechselzeit aus der Anzahl der Aus-, Einsteiger und Platzmacher abgeschätzt werden?"' (Auswertung in Abschnitt \ref{Zeiten}) \label{item:Fahrgastwechselzeiten}
 \item "`Welche Verhaltensweisen und Merkmale von Fahrgästen beeinflussen die benötigte Zeit für den Fahrgastwechsel?"' \label{item:Verhalten}
 	\begin{enumerate}
 		\item "`Beeinflussen die Verhaltensweisen "`früher Einsteigen"' und "`Im Weg Stehen"' die Fahrgastwechselzeiten?"' (Beschreibung der Verhaltensweisen in \ref{Datenerhebung}, Auswertung in \ref{Verhalten}) \label{item:Verhalten,Verhalten}
 		\item "`Beeinflussen die Merkmale "`sperrig"', "`langsam"' und "`abgelenkt"' der Fahrgäste die Fahrgastwechselzeit?"' (Beschreibung der Merkmale in \ref{Datenerhebung}, Auswertung in \ref{Verhalten}) \label{item:Vehalten,Merkmale}
 	\end{enumerate}
 \item "`Welche Entscheidungen werden beim Prozess des Fahrgastwechsel getroffen?"' (Auswertung in Kapitel \ref{Modell}) \label{item:Entscheidungen}
\end{enumerate}
Zudem kamen noch folgende Forschungsfragen auf:
\begin{enumerate}
\setcounter{enumi}{3}
	\item "`Gibt einen Zusammenhang zwischen der Anzahl der Aussteiger und dem Vorkommen von Platzmachern?"' (Auswertung in \ref{Verhalten})  \label{item:Platzmacher}
	\item "`Kann aus dem Filmmaterial gewonnen werden, welche Gruppengrößen im Fahrgastwechsel auftreten und wie viel Prozent der Personen sich in einer Gruppe mit einer bestimmten Größe befinden?"' Die Größe einer Gruppe wird definiert durch die Anzahl ihrer Mitglieder. Mitglied einer Gruppe ist, wer zu mindestens einem anderen Mitglied der Gruppe Kontakt hat, \dahe er mit einem Mitglied der Gruppe redet oder auf ein Mitglied der Gruppe wartet. \label{item:Goupsize} 
	\item Auf den Videos wurden Verhaltensweisen, erkannt die als "`defensiv"' oder "`aggressiv"' bezeichnet werden können. Dazu gehören unter anderem Verhaltensweisen wie das in den Weg anderer Personen laufen oder mit dem Einsteigen warten bis alle aus dem Zug sind, auch wenn genug Platz zum Einsteigen wäre. Zeigte eine Person eine oder mehrere dieser Verhaltensweisen wurde sie als "`defensiv"' oder "`aggressiv"' bezeichnet. Zeigte sie keine dieser Verhaltensweisen galt die Person als "`normal"'. Wie die Personen in diese Typen (aggressiv, defensiv und normal) eingeteilt wurden wird in \ref{Tabelle der Typen} genauer erklärt. Um herauszufinden ob die Einteilung in Typen für das Modell interessant ist soll die Frage: "`Gibt es einen signifikanten Anteil an defensiven und aggressiven Personen?"' geklärt werden.(Auswertung in Abschnitt \ref{Typen}) \label{item:Typen,Typen}
	\begin{enumerate}
		\item Nachdem mir auffiel, dass die unterschiedlichen Typen von Aussteigern unterschiedlich lange warten, bis sie aussteigen, kam zudem die Frage auf: "`Liegen die Startzeiten der Aussteiger für die unterschiedlichen Typen in unterschiedlichen Intervallen?"' (Auswertung in Abschnitt \ref{Startzeiten}) \label{item:Typen,Startzeiten}
	\end{enumerate} 
\end{enumerate}